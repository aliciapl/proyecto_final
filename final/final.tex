
\documentclass[]{article}

%Packages
\usepackage[spanish]{babel}
\usepackage[utf8]{inputenc}
%Fórmulas
\usepackage{amsmath, amsthm, amssymb}
%Imágenes
\usepackage{graphicx}

%Keywords en español
\newenvironment{keywords}
{\par\noindent\small\textbf{\keywordsname:}}
{\par}
\addto\captionsspanish{\def\keywordsname{Palabras Claves}}






\begin{document}
\selectlanguage{spanish}

\title{Patogénesis de \textit{Staphylococcus}}

\author{Alicia Isabel Pérez Lorente}



\maketitle              


\begin{abstract} https://github.com/aliciapl/proyecto\_final \\
	En este trabajo se presenta ..

\begin{keywords}
	Palabras
\end{keywords}


\end{abstract}


\section{Introducción}

\section{Estado del arte}
En el trabajo de Lina et al. \cite{Lina1999} y en el trabajo de Hiramatsu et al. \cite{Hiramatsu1997}...
\section{Imágenes y tablas / Evaluación}


\begin{figure}[!h]
	\centering
	\includegraphics[width=0.8\textwidth]{images/prueba.png}
	\caption{ Imagen de prueba. } \label{fig1}
\end{figure}

\begin{table}[!h]
	\centering
	\caption{Tabla de prueba}
	\label{tab1}
	
		\begin{tabular}{c|c|c|}
			\cline{2-3}
			
			& \textbf{Evaluación 1} & \textbf{Evaluación 2} \\ \hline
			
			\multicolumn{1}{|c|}{\textbf{Cepa 1}} &   uno                     &          dos             \\ \hline
			\multicolumn{1}{|c|}{\textbf{Cepa 2}} &   tres                    &          cuatro             \\ \hline
			\multicolumn{1}{|c|}{\textbf{Cepa 3}} &      cinco                 &          seis             \\ \hline
			\multicolumn{1}{|c|}{\textbf{Cepa 4}} &         siete              &      ocho                 \\ \hline
		\end{tabular}%
	
\end{table}


\section{Fórmulas / Evaluación}

$m=\frac{y_2-y_1}{x_2-x_1}$


\bibliographystyle{unsrt}
\bibliography{biblio}



\end{document}
